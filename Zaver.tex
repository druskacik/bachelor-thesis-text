V závere je potrebné v stručnosti zhrnúť dosiahnuté výsledky vo vzťahu k stanoveným cieľom. 
Vzhľadom na to, že v práci už boli definované všetky potrebné pojmy, možno byť v závere omnoho konkrétnejší ako v úvode. Pri argumentovaní o splnení cieľov je žiaduce odkazovať na konkrétne očíslované alebo inak pomenované časti práce (odseky, algoritmy, metódy, vzorce, vzťahy, vety a ich dôkazy a pod.).
