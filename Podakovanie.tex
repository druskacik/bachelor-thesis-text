

\vglue0pt
\vfill
\thispagestyle{empty}
\paragraph{Poďakovanie}[nepovinné]
Na tomto mieste môže byť vyjadrenie poďakovania napr. vedúcemu práce resp. konzultantom za pripomienky a odbornú pomoc pri vypracovaní práce. Vyjadrenie poďakovania v prípade využitia inej práce (pomoci) sa uskutočňuje formou citácie na konci hlavného textu práce a odkazy na citáciu sa musia uviesť aj na zodpovedajúcich miestach v texte. 

{\it Možno tu však ďakovať konkrétnym osobám za pomoc pri vytlačení práce, za pomoc pri skontrolovaní pravopisu, za finačnú a morálnu podporu rodičom a pod. Vždy treba uviesť konkrétny druh pomoci, treba sa vyhnúť všeobecnému poďakovaniu za pomoc osobám iným ako je vedúci práce. Budí to podozrenie,  že ste prácu nevypracovali samostatne. 

Príklad poďakovania (K. Pokorná 2010):} \\
Touto cestou sa chcem poďakovat svojej vedúcej bakalárskej práce Doc. RNDr. Margaréte Halickej, CSc. za ochotu, pomoc, odborné rady a podnetné pripomienky, ktoré mi pomohli pri písaní tejto práce. Ďakujem aj svojej rodine a priateľom za ich trpezlivosť a podporu.
