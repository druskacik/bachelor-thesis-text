Vykonávanie experimentov a prieskumov patrí medzi populárne mechanizmy, ako potvrdiť či vyvrátiť naše domnienky, alebo odhaliť nové, neočakávané skutočnosti. 
V súčasnosti, vďaka nebývalému rozvoju v oblasti výpočtovej techniky a spracovania dát, je tento trend silnejší, než kedykoľvek v minulosti.

Predtým, než sa experimentátor pustí do zberu dát a ich analýzy, je potrebné experiment navrhnúť tak, aby jeho výsledky dávali dostatočný priestor na 
logickú a nespochybniteľnú analýzu. V ideálnom prípade chceme, aby miera informácie, ktorú zozbierané dáta ponúkajú, bola čo najväčšia.

V našej práci sa budeme zaoberať návrhom tzv. riadkovo-stĺpcových experimentov, v ktorých kombinácie dvoch kvalitatívnych faktorov spájame s jedným z dvoch typov ošetrení. 
Tieto návrhy budeme reprezentovať binárnymi maticami, ktoré, ako ukážeme, vieme transformovať na lineárny regresný model. 
Teóriou lineárnej regresie sa zaoberá množstvo článkov a publikácií, v našej práci budeme čerpať najmä z \cite{pazman, yan}.

Spolu s teóriou lineárnej regresie uvedieme v teoretickej časti niektoré dôležité poznatky z lineárnej algebry, ktoré budeme využívať v celej práci. 
V tretej kapitole v krátkosti opíšeme, ako sme pri našom výskume postupovali.

V štvrtej kapitole budeme skúmať riadkovo-stĺpcové návrhy. Cieľom je analyzovať kontrast medzi dvoma typmi ošetrení. 
Uvedieme, ktorá trieda návrhov ponúka dostatočnú informáciu na odhadnutie tohto kontrastu, a tiež pre ktoré z nich je tento odhad najpresnejší. 
Na záver v krátkosti ukážeme, ako možno riadkovo-stĺpcové návrhy reprezentovať bipartitným grafom.