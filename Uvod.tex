
Hlavnú textovú časť záverečnej práce tvorí: úvod, jadro, záver, zoznam použitej literatúry.

V úvode autor stručne a výstižne charakterizuje stav poznania alebo praxe v oblasti, ktorá je predmetom záverečnej práce a oboznamuje čitateľa s významom, cieľmi a zámermi práce. Autor v úvode zdôrazňuje, prečo je práca dôležitá a prečo sa rozhodol spracovať danú tému. 


Rozsah úvodu je  1 až 1,5 strany.  Úvod spravidla obsahuje: 
\begin{itemize}
\item vymedzenie problematiky BP v kontexte aplikácií matematiky;
\item zdôvodnenie aktuálnosti danej témy;
\item charakterizácia stavu poznania alebo praxe (odkazy na literatúru);
\item nastolenie problémov, ktoré chce autor v BP riešiť;
\item vytýčenie cieľov, ktoré majú byť v BP dosiahnuté;
\item uvedenie použitých metód a postupov riešenia;
\item stručný náčrt obsahu jednotlivých kapitol BP.
\end{itemize}
