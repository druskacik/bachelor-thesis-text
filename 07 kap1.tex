\section{Maticová príprava}
\label{matrix algebra}

V našej práci budeme často pracovať s poznatkami z lineárnej algebry, preto v krátkosti zhrnieme niektoré najdôležitejšie z nich.

\begin{defin}
Nech $A$ je ľubovoľná matica typu $m \times n$. Potom symbolom $\mathcal{M}(A)$ označujeme stĺpcový priestor matice $A$, t. j.
\begin{center}
$
\mathcal{M}(A) = \{ Au : u \in \mathbb{R}^n \}
$
\end{center}
Symbolom $\mathcal{K}(A)$ označujeme jadro matice (kernel), t. j.
\begin{center}
$
\mathcal{K}(A) = \{ v \in \mathbb{R}^n : Av = 0 \}
$.
\end{center}
\end{defin}

\begin{lema}
Platí
\begin{center}
$
\mathcal{K}(A) = [\mathcal{M}(A)]^{\perp} \equiv \{ v \in \mathbb{R}^n : {\forall}_{u \in \mathcal{M}(A)} u^T v = 0 \}
$
\end{center}
\end{lema}

\begin{dokaz}


\begin{center}
$
x \in \mathcal{K}(A^T) \Leftrightarrow A^T x = 0 \Leftrightarrow {\forall}_v v^T A^T x = 0 \Leftrightarrow {\forall}_v (Av)^Tx = 0 \Leftrightarrow x \in [\mathcal{M}(A)]^{\perp}
$
\end{center}
\end{dokaz}

\begin{lema}
\label{o stlpcovych priestoroch}
Nech $A$ je ľubovoľná matica typu $m \times n$. Potom
\begin{center}
$\mathcal{M}(A^T) = \mathcal{M}(A^T A)$
\end{center}
\end{lema}

\begin{dokaz}
Nech $z \in \mathcal{M}(A^T A)$. Potom existuje také $v$, že $z = A^T A v = A^T u$ pre $u = A v$.
Preto $z \in \mathcal{M}(A^T)$ a teda $\mathcal{M}(A^T A) \subseteq \mathcal{M}(A^T)$.

Teraz nech $z \in \mathcal{K}(A^T A)$. Potom:

\begin{center}
$
A^T A z = (0, 0, \ldots, 0)^T \Rightarrow z^T A^T A z = 0 = (Az)^T Az = ||Az|| \Rightarrow Az = 0
$
\end{center}

Preto $z \in \mathcal{K}(A)$, a teda $\mathcal{K}(A^T A) \subseteq \mathcal{K}(A)$.
Z lineárnej algebry ale vieme, že nulový priestor matice je kolmý doplnok jej riadkového priestoru,
preto predošlý výraz vieme rozšíriť:
\begin{center}
$
[\mathcal{M}(A^T A)]^{\perp} = \mathcal{K}(A^T A) \subseteq \mathcal{K}(A) = [\mathcal{M}(A^T)]^{\perp}
$
\end{center}

Odstránením $\perp$ sa inklúzia obráti, čo spolu s opačnou inklúziou vyššie dáva výsledok $\mathcal{M}(A^T) = \mathcal{M}(A^T A)$.
\end{dokaz}

\begin{defin}
Nech $A$ je ľubovoľná matica. Potom matica $A^-$ taká, že
\begin{center}
$A A^- A = A$
\end{center}
sa nazýva $g$-inverzia (alebo pseudoinverzia) matice $A$.
\end{defin}

\begin{com}
Ak $A^{-1}$ neexistuje, tak pre maticu $A$ existuje nekonečný počet $g$-inverzií.
\end{com}

Pseudoinverzie použijeme neskôr v našej práci, keď budeme skúmať, 
či vektor patrí do stĺpcového priestoru matice. Na to nám poslúži nasledovná veta.

\begin{theorem}
\label{veta3}
Nech rovnica $Ax = y$ má riešenie a nech $A^-$ je ľubovoľná $g$-inverzia. Potom $A^- y$ je riešením tejto rovnice.
\end{theorem}

\begin{dokaz}
Keďže $Ax = y$ má riešenie, tak existuje také $x_0$, že $A x_0 = y$. Potom 
\begin{center}
$A (A^- y) = A(A^- A x_0) = A x_0 = y$
\end{center}
\end{dokaz}

\subsection{Projekčné matice}

Pri práci s lineárnym regresným modelom sa nám zídu poznatky o projekčných maticiach, napr. pri metóde najmenších štvorcov.

\begin{defin}
\label{linearny projektor}
Nech $\mathcal{V} \subset \mathbb{R}^n$ je lineárny priestor. Potom matica $P$ je lineárny projektor na $\mathcal{V}$ práve vtedy,
keď platia nasledovné podmienky:
\begin{enumerate}[label=\emph{\alph*})]
  \item
    $
    {\forall}_{x \in \mathbb{R}^n} Px \in \mathcal{V}
    $
  \item
    $
    {\forall}_{v \in \mathcal{V}} Pv = v
    $
\end{enumerate}
\end{defin}

Z vlastností a) a b) vyplýva, že $P^2 = P$, t. j. matica $P$ je idempotentná.
Poznamenávame, že aj naopak, ak matica $P$ je idempotentná a definujeme $V = \mathcal{M}(P)$, tak $P$ má vlastnosti a) a b) z predošlej definície.

Takto definovaný projektor nemusí byť ortogonálny, t. j. nezobrazuje vektory na k nim najbližší vektor z priestoru $V$.
Formálne ortogonalitu definujeme nasledovne.

\begin{defin}
\label{ortogonalny projektor}
Projektor $P$ na lineárny podpriestor $V$ je ortogonálny vzhľadom na skalárny súčin $\langle \cdot,\cdot \rangle$,
ak $\langle x, Py \rangle = \langle Px, y \rangle$ pre všetky $x, y \in \mathbb{R}^n$.
\end{defin}

\begin{theorem}
Nech $F \in \mathbb{R}^{m \times m}$, $m > n$ má plnú hodnosť a nech $\rangle a, b \langle = a^T b$ je skalárny súčin v $\mathbb{R}^m$.
Potom ortogonálny projektor na $\mathcal{M}(F)$ je
\begin{center}
$
P = F(F^T F)^{-1} F^T
$
\end{center}
\end{theorem}

\begin{dokaz}
Ukážeme, že $P$ spĺňa vlastnosti z definície lineárneho projektora, aj podmienku ortogonality z definície \ref{ortogonalny projektor}.
\begin{center}
$
{\forall}_{x \in \mathbb{R}^m} Px = F(F^T F)^{-1} F^T x = Fu \in \mathcal{M}(F)
$
\end{center}
\begin{center}
$
{\forall}_{l = Fu \in \mathcal{M}(F)} Pl = P(Fu) = F(F^T F)^{-1} F^T F u = Fu = l
$
\end{center}
\begin{center}
$
{\forall}_{x, y \in \mathbb{R}^m} \langle x, Py \rangle = x^T F(F^T F)^{-1} F^T y = x^T [F (F^T F)^{-1} F^T]^T y = x^T P^T y = \langle Px, y \rangle
$
\end{center}
\end{dokaz}

\begin{theorem}
\label{ortogonalny projektor na maticu}
Predchádzajúca veta platí aj pre maticu $F$, ktorá nemá plnú hodnosť,
ak namiesto inverzie $(F^T F)^{-1}$ použijeme ľubovoľnú $g$-inverziu $(F^T F)^{-}$.
\end{theorem}