\section{Môj model}
\label{my model}

Cieľom tejto práce je skúmať experiment reprezentovaný binárnou maticou.
Možná reprezentácia nášho modelu pri matici rozmeru $m \times n$ je takáto:

Uvažujeme model s dvomi kvalitatívnymi faktormi $A$ a $B$, 
pričom faktor $A$ má $m$ úrovní a faktor $B$ $n$ úrovní. 
Pre každú kombináciu faktorov experimentátor zvolí jedno z dvojice ošetrení (angl. treatment).
Budeme uvažovať len aditívny model bez interakcií.

(Harmanov text == TODO: prepísať)Napríklad faktor $A$ môže označovať dobrovoľníka (t.j. máme $m$ dobrovoľníkov) 
a faktor $B$ účinnú látku, ktorú nazveme liečivo (t.j. máme $n$ liečiv). 
Ošetrenie $0$ reprezentuje podanie liečiva orálnym spôsobom a ošetrenie $1$ vnútrožilne).
Účinok, ktorý po čase nameriame na dobrovoľníkovi, 
závisí aditívne od efektu samotného dobrovoľníka (jeho predispozícií), 
efektu liečiva a efektu ošetrenia. 
Všetky tieto efekty uvažujeme ako neznáme parametere modelu.

Návrh teda možno reprezentovať binárnou maticou 
(alebo bipartitným grafom, čo opíšeme neskoršie vnašej práci).

Cieľom experimentu je odhadnúť rozdiel medzi dvoma efektmi (treamentami).
Cieľom hľadania optimálneho modelu je nájsť taký návrh, pri ktorom
rozptyl Gauss-Markovovho odhadu rozdielu medzi efektmi dvojice ošetrení (t.j. odhadu kontrastu ošetrení) bude čo najmenší.
(Použijeme pri tom Gaussovu-Markovovu vetu z predošlej kapitoly.)

Pre maticu typu $m \times n$ dostaneme teda $mn$ vzťahov tvaru:

\begin{center}
$
y_{ij} = a_i + b_j + t + e_{ij}
$
\end{center}

$i = 1, \ldots, m$; $j = 1, \ldots, n$; $t$ je $t_0$ alebo $t_1$ podľa hodnoty na $ij$-tej pozícii matice a $e_{ij}$ je chyba.
Budeme odhadovať rozdiel medzi $t_0$ a $t_1$.

Z návrhu modelu v podobe binárnej matice tvaru $m \times n$ vytvoríme lineárny regresný model
s maticou tvaru $mn \times (m + n + 2)$, pretože máme $mn$ meraní závislých od $m + n + 2$ premenných.

Príklad:

Maticu návrhu experimentu

\begin{align}
\label{model matrix}
\begin{bmatrix}
1 & 0 & 1 \\
0 & 1 & 0 \\
1 & 0 & 1
\end{bmatrix}
\end{align}

prepíšeme ako (TODO: popíš tými strapatými zátvorkami čo znamená čo):

\begin{align}
\label{linear regression matrix}
\begin{bmatrix}
1 & 0 & 0 & 1 & 0 & 0 & 1 & 0 \\
1 & 0 & 0 & 0 & 1 & 0 & 0 & 1 \\
1 & 0 & 0 & 0 & 0 & 1 & 1 & 0 \\
0 & 1 & 0 & 1 & 0 & 0 & 0 & 1 \\
0 & 1 & 0 & 0 & 1 & 0 & 1 & 0 \\
0 & 1 & 0 & 0 & 0 & 1 & 0 & 1 \\
0 & 0 & 1 & 1 & 0 & 0 & 1 & 0 \\
0 & 0 & 1 & 0 & 1 & 0 & 0 & 1 \\
0 & 0 & 1 & 0 & 0 & 1 & 1 & 0
\end{bmatrix}
\end{align}

Označme vo všeobecnosti maticu modelu (\ref{model matrix}) $B$ a maticu lineárneho zobrazenia (\ref{linear regression matrix}) $X$. 
Predpokladáme lineárne vzťahy, preto dostávame lineárnu regresiu

\begin{center}
$
y = X b + e
$
\end{center}

kde $y$ je vektor dát nameraných v experimente, $b$ je vektor neznámych koeficientov a 
$e = (e_{11}, e_{12}, e_{13}, \ldots, e_{m1}, \dots, e_{mn})^T$ je vektor chýb merania. 
Uvedomme si, že v našej práci nebudeme pracovať s konkrétnym vektorom $y$; 
snažíme sa totiž navrhnúť maticu modelu $B$ (a z nej vyplývajúcu maticu $X$) ešte pred tým, než experiment vykonáme.

Naším cieľom teraz bude odhadnúť rozdiel medzi efektmi $t_0$ a $t_1$, resp. zistiť, 
aká môže byť disperzia tohto odhadu pri danej matici návrhu $B$. 
Modely, pre ktoré bude možná disperzia najmenšia, budeme považovať za optimálne.

V danej lineárnej regresii $y = X b + e$ teda nebudeme odhadovať celý vektor $b$, 
ale len jeho lineárnu kombináciu $h^T b$, kde $h$ je vektor rozmeru $m + n + 2$, 
ktorý má na $m + n$ miestach $0$ a na zvyšných dvoch miestach $+1$ a $-1$, ktoré zodpovedajú efektom $t_0$ a $t_1$. 
Z algoritmu, ktorým sme z matice návrhu $B$ vytvorili maticu lineárnej regresie $X$, 
vyplýva, že vektor $h$ má tvar $(0, 0, 0, \ldots, +1, -1)^T$.

Na zistenie, či $h^T b$ je odhadnuteľné, použijeme vetu (\ref{veta1}), 
a na následné nájdenie minimálneho rozptylu Gauss-Markovovho odhadu rozdielu medzi efektmi 
použijeme Gaussovu-Markovovu vetu (\ref{gauss-markov}).