\section{Môj model}
\label{my model}

Cieľom tejto práce je skúmať experiment reprezentovaný binárnou maticou.
Možná reprezentácia nášho modelu pri matici rozmeru $m \times n$ je takáto:

Uvažujeme model s dvomi kvalitatívnymi faktormi $A$ a $B$, 
pričom faktor $A$ má $m$ úrovní a faktor $B$ $n$ úrovní. 
Pre každú kombináciu faktorov experimentátor zvolí jedno z dvojice ošetrení (angl. treatment).
Budeme uvažovať len aditívny model bez interakcií.

(Harmanov text == TODO: prepísať)Napríklad faktor $A$ môže označovať dobrovoľníka (t.j. máme $m$ dobrovoľníkov) 
a faktor $B$ účinnú látku, ktorú nazveme liečivo (t.j. máme $n$ liečiv). 
Ošetrenie $0$ reprezentuje podanie liečiva orálnym spôsobom a ošetrenie $1$ vnútrožilne).
Účinok, ktorý po čase nameriame na dobrovoľníkovi, 
závisí aditívne od efektu samotného dobrovoľníka (jeho predispozícií), 
efektu liečiva a efektu ošetrenia. 
Všetky tieto efekty uvažujeme ako neznáme parametere modelu.

Návrh teda možno reprezentovať binárnou maticou 
(alebo bipartitným grafom, čo opíšeme neskoršie vnašej práci).

Cieľom experimentu je odhadnúť rozdiel medzi dvoma efektmi (treamentami).
Cieľom hľadania optimálneho modelu je nájsť taký návrh, pri ktorom
rozptyl Gauss-Markovovho odhadu rozdielu medzi efektmi dvojice ošetrení (t.j. odhadu kontrastu ošetrení) bude čo najmenší.
(Použijeme pri tom Gaussovu-Markovovu vetu z predošlej kapitoly.)

Pre maticu typu $m \times n$ dostaneme teda $mn$ vzťahov tvaru:

\begin{center}
$
y_{ij} = a_i + b_j + t + e_{ij}
$
\end{center}

$i = 1, \ldots, m$; $j = 1, \ldots, n$; $t$ je $t_0$ alebo $t_1$ podľa hodnoty na $ij$-tej pozícii matice a $e_{ij}$ je chyba.
Budeme odhadovať rozdiel medzi $t_0$ a $t_1$.

Z návrhu modelu v podobe binárnej matice tvaru $m \times n$ vytvoríme lineárny regresný model
s maticou tvaru $mn \times (m + n + 2)$, pretože máme $mn$ meraní závislých od $m + n + 2$ premenných.

Príklad:

Maticu návrhu experimentu

\begin{align}
\label{model matrix}
\begin{bmatrix}
1 & 0 & 1 \\
0 & 1 & 0 \\
1 & 0 & 1
\end{bmatrix}
\end{align}

prepíšeme ako (TODO: popíš tými strapatými zátvorkami čo znamená čo):

\begin{align}
\label{linear regression matrix}
\begin{bmatrix}
1 & 0 & 0 & 1 & 0 & 0 & 1 & 0 \\
1 & 0 & 0 & 0 & 1 & 0 & 0 & 1 \\
1 & 0 & 0 & 0 & 0 & 1 & 1 & 0 \\
0 & 1 & 0 & 1 & 0 & 0 & 0 & 1 \\
0 & 1 & 0 & 0 & 1 & 0 & 1 & 0 \\
0 & 1 & 0 & 0 & 0 & 1 & 0 & 1 \\
0 & 0 & 1 & 1 & 0 & 0 & 1 & 0 \\
0 & 0 & 1 & 0 & 1 & 0 & 0 & 1 \\
0 & 0 & 1 & 0 & 0 & 1 & 1 & 0
\end{bmatrix}
\end{align}

Označme vo všeobecnosti maticu modelu (\ref{model matrix}) $B$ a maticu lineárneho zobrazenia (\ref{linear regression matrix}) $X$. 
Predpokladáme lineárne vzťahy, preto dostávame lineárnu regresiu

\begin{center}
$
y = X b + e
$
\end{center}

kde $y$ je vektor dát nameraných v experimente, $b$ je vektor neznámych koeficientov a 
$e = (e_{11}, e_{12}, e_{13}, \ldots, e_{m1}, \dots, e_{mn})^T$ je vektor chýb merania. 
Uvedomme si, že v našej práci nebudeme pracovať s konkrétnym vektorom $y$; 
snažíme sa totiž navrhnúť maticu modelu $B$ (a z nej vyplývajúcu maticu $X$) ešte pred tým, než experiment vykonáme.

Naším cieľom teraz bude odhadnúť rozdiel medzi efektmi $t_0$ a $t_1$, resp. zistiť, 
aká môže byť disperzia tohto odhadu pri danej matici návrhu $B$. 
Modely, pre ktoré bude možná disperzia najmenšia, budeme považovať za optimálne.

V danej lineárnej regresii $y = X b + e$ teda nebudeme odhadovať celý vektor $b$, 
ale len jeho lineárnu kombináciu $h^T b$, kde $h$ je vektor rozmeru $m + n + 2$, 
ktorý má na $m + n$ miestach $0$ a na zvyšných dvoch miestach $+1$ a $-1$, ktoré zodpovedajú efektom $t_0$ a $t_1$. 
Z algoritmu, ktorým sme z matice návrhu $B$ vytvorili maticu lineárnej regresie $X$, 
vyplýva, že vektor $h$ má tvar $(0, 0, 0, \ldots, +1, -1)^T$.

Na zistenie, či $h^T b$ je odhadnuteľné, použijeme vetu (\ref{veta1}), 
a na následné nájdenie minimálneho rozptylu Gauss-Markovovho odhadu rozdielu medzi efektmi 
použijeme Gaussovu-Markovovu vetu (\ref{gauss-markov}).

\subsection{Odhadnuteľnosť $h^T b$}

Pokúsime sa preskúmať, kedy matica návrhu umožňuje odhadnuteľnosť hodnoty $h^T b$ pre vyššie spomenuté $h$. 
Situáciu preskúmame do rozmeru $4 \times 5$ a na základe našich zistení vyslovíme hypotézu pre všeobecný prípad $m \times n$.

Postupovať budeme nasledovne: vyberieme si malý rozmer a pre všetky matice daného rozmeru zistíme, 
či $h^T b$ bude alebo nebude odhadnuteľné, a to nasledovným spôsobom. 
Z danej matice návrhu $B$ vytvoríme maticu lineárnej regresie $X$ spôsobom spomenutým vyššie. 
Hodnota $h^T b$ bude na základe vety (\ref{veta1}) odhadnuteľná práve vtedy, 
keď vektor $h = (0, 0, 0, \ldots, +1, -1)^T$ patrí do stĺpcového priestoru matice $X^T X$.

Označme $M = X^T X$, túto maticu budeme nazývať informačnou maticou. 
Ak $h$ patrí do stĺpcového priestoru $M$, potom existuje taký vektor $v$, že $M v = h$. 
Na základe vety (\ref{veta3}) vieme, že ak toto riešenie existuje, tak sa rovná $M^- h$, pričom zvolená $g$-inverzia môže byť ľubovoľná.
Preto na zistenie, či $h$ patrí do stĺpcového priestoru $M = X^T X$ stačí overiť platnosť rovnosti $M(M^- h) = h$. 

Demonštrujme algoritmus na maticiach rozmeru $3 \times 3$. Existuje $2^9 = 512$ binárnych matíc typu $3 \times 3$, 
a keď sme na nich rozbehli daný algoritmus, dospeli sme k zisteniu, že pri $12$ z nich hodnota $h^T b$ NIE JE odhadnuteľná.
To znamená, že optimálne matice návrhu budeme hľadať v zvyšných $500$ maticiach.

Ako vyzerajú matice, pri ktorých $h^T b$ nie je odhadnuteľné? Zoznam všetkých sa nachádza v Prílohe A, tu uvedieme $3$ z nich:

\begin{center}
$
\begin{bmatrix}
1 & 1 & 1 \\
0 & 0 & 0 \\
1 & 1 & 1 \\
\end{bmatrix}
$, 
$
\begin{bmatrix}
0 & 0 & 1 \\
0 & 0 & 1 \\
0 & 0 & 1 \\
\end{bmatrix}
$, 
$
\begin{bmatrix}
0 & 0 & 0 \\
0 & 0 & 0 \\
0 & 0 & 0 \\
\end{bmatrix}
$
\end{center}

Týmto spôsobom sme pri našom výskume získali zoznam matíc, pri ktorých $h^T b$ nie je odhadnuteľné, až do rozmeru $4 \times 5$.
Zoznam všetkých sa nachádza v GitHub repozitári spomenutom v úvode. 
Na základe podobnosti matíc daného typu sme prišli s nasledovnou hypotézou.

\begin{hypoteza}
$h^T b$ je odhadnuteľné práve vtedy, keď v matici návrhu $B$ existuje aspoň jeden riadok a aspoň jeden stĺpec taký,
ktorý má aspoň jednu $0$ a aspoň jednu $1$.
\end{hypoteza}

\begin{dokaz}
Na základe vety (\ref{veta1}) vieme, že $h^T b$ je odhadnuteľné práve vtedy, keď $h \in \mathcal{M}(X^T)$
(keď $h$ patrí do riadkového priestoru $X$),
preto s danými podmienkami môžeme pracovať ekvivalentne.

$\boxed{\implies}$ Nech $h = (0, \ldots, +1, -1) \in \mathcal{M}(X^T)$. Použijeme dôkaz sporom. 
Nech v matici $B$ typu $m \times n$ neexistuje taký riadok, ktorý má aspoň jednu $0$ a aspoň jednu $1$,
t. j všetky riadky obsahujú buď samé $0$ alebo samé $1$.
Ako vyzerá matica $X$?

Pozrime sa na riadky matice $X$ zodpovedajúce prvému riadku matice $B$.
Sú to práve tie riadky, ktoré majú na prvej pozícii $1$, pričom si uvedomme, že všetky ostatné riadky majú na prvej pozícii $0$.

% TODO
\begin{center}
$
x_1 = (1, 0, \ldots, 1, 0, \ldots, 0, 1, 0)
$,
\end{center}
\begin{center}
$
x_2 = (1, 0, \ldots, 0, 1, \ldots, 0, 1, 0)
$,
\end{center}
\begin{center}
$\vdots$
\end{center}
\begin{center}
$
x_n = (1, 0, \ldots, 0, 0, \ldots, 1, 1, 0)
$
\end{center}

Keďže $h \in \mathcal{M}(X^T)$, existuje taká lineárna kombinácia riadkov $X$, ktorá dokopy dáva vektor $h$.
Aké môžu mať v tejto lineárnej kombinácii zastúpenie riadky $x_1$, ..., $x_n$?
Označme postupne $a_1$, ..., $a_n$ koeficienty vektorov $x_1$, ..., $x_n$ v lineárnej kombinácii riadkov $X$, ktorá dáva vektor $h$.
Keďže $x_1$, ..., $x_n$ sú jediné riadky matice $X$ s $1$ na prvej pozícii a vektor $h$ má na prvej pozícii $0$,
musí platiť $\sum_{i = 1}^n a_i = 0$.

Z predpokladu, že každý z riadkov matice $B$ obsahuje buď samé $0$ alebo samé $1$, vyplýva,
že každý z vektorov $x_1$, ..., $x_n$ má rovnaké posledné dve hodnoty, a to buď $(0, 1)$ alebo $(1, 0)$.
Z toho spolu s rovnosťou $\sum_{i = 1}^n a_i = 0$ vyplýva, že lineárna kombinácia $\sum_{i = 1}^n a_i x_i$
bude mať na posledných dvoch miestach hodnoty $(0, 0)$, a teda nijako "neprispeje" do vektoru $h$.

Analogicky môžeme postupovať pre všetky riadky matice $B$.
Týmto spôsobom pokryjeme všetky riadky matice $X$, vďaka čomu dospejeme k záveru,
že z riadkov $X$ nie je možné lineárnou kombináciou zostrojiť vektor $h$.
To nám dáva spor s predpokladom $h \in \mathcal{M}(X^T)$.
Preto nemôže platiť, že všetky riadky matice $B$ obsahujú buď samé $0$ alebo samé $1$,
a platí opačné tvdenie, t.j v $B$ existuje aspoň jeden riadok taký, ktorý obsahuje $0$ aj $1$.

Rovnakou úvahou pre stĺpce $B$ dospejeme k záveru,
že $B$ musí takisto obsahovať aspoň jeden stĺpec obsahujúci $0$ aj $1$.

$\boxed{\impliedby}$ Nech matica $B$ je taká, že existuje aspoň jeden riadok a aspoň jeden stĺpec také, že majú $0$ aj $1$.
Zostrojíme takú lineárnu kombináciu riadkov $X$, ktorá sa bude rovnať $h$.

Nech riadok, ktorý má $0$ aj $1$, je $i$-ty v poradí a stĺpec s rovnakou vlastnosťou je $j$-ty v poradí.
Bez ujmy na všeobecnosti, nech na $ij$-tom mieste matice $B$ sa nachádza $0$, teda $B_{ij} = 0$.

Potom v $i$-tom riadku $B$ sa určite nachádza hodnota $B_{ik} = 1$ a v $j$-tom stĺpci sa nachádza hodnota $B_{lj} = 1$,
pričom, samozrejme, $k \neq j$ a $l \neq i$.
Označme $x_{ij}, x_{ik}, x_{lj}, x_{lk}$ riadky matice $X$ prislúchajúce prvkom $B_{ij}, B_{ik}, B_{lj}, B_{lk}$ matice $B$. Potom:

\begin{center}
$
x_{ij} - x_{ik} - x_{lj} + x_{lk} = (0, 0, 0, \ldots, 0, N_1, N_2)
$
\end{center}
je vektor, ktorý má na prvých $m + n$ miestach $0$, a na posledných dvoch hodnoty $N_1$ a $N_2$, ktoré zatiaľ necháme bokom.
Prečo má daný vektor na prvých $m + n$ miestach $0$?
Vo všeobecnosti má riadok $x_{rs}$ matice $X$ prislúchajúci prvku $B_{rs}$ matice $B$ na prvých $m + n$ miestach dve $1$,
jednu prislúchajúcu riadku, druhú stĺpcu matice $B$. Konkrétne, riadok $x_{rs}$ má $1$ na $r$-tom mieste a $(m + s)$-tom mieste.

Súčet $x_{ij} + x_{lk}$ má teda štyri jednotky na miestach $i, j, m + j, m + k$.
Súčet $x_{ik} + x_{lj}$ má jednotky na tých istých miestach, 
preto $x_{ij} - x_{ik} - x_{lj} + x_{lk} = x_{ij} + x_{lk} - (x_{ik} + x_{lj})$ má na prvých $m + n$ miestach $0$.

Takto to vyzerá, keď zanedbáme stĺpce so samými nulami:
\[
\begin{split}
+(1, 0, 1, 0) \\
-(1, 0, 0, 1) \\
-(0, 1, 1, 0) \\
+(0, 1, 0, 1) \\
=(0, 0, 0, 0) \\
\end{split}
\]

Ako vyzerá chvost $(N_1, N_2)$? Keďže $B_{ij} = 0$ a $B_{ik} = B_{lj} = 1$,
výraz $x_{ij} - x_{ik} - x_{lj}$ nám vytvorí chvost $(1, -2)$ (prípadne $(-2, 1)$, na tom ale nezáleží).
Potom na základe toho, či na mieste $B_{lk}$ matice návrhu bola $0$ alebo $1$,
nám výraz $x_{ij} - x_{ik} - x_{lj} + x_{lk}$ dá na posledných dvoch miestach $(2, -2)$ alebo $(1, -1)$.
Vo všetkých prípadoch vektor $h$ dostaneme ihneď, prípadne po prenásobení konštantou.
Našli sme teda lineárnu kombináciu riadkov $X$, ktorá nám dala vektor $h$, preto $h \in \mathcal{M}(X^T)$. $\square$

\end{dokaz}