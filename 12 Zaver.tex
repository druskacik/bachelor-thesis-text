V práci sme skúmali tzv. riadkovo-stĺpcové návrhy štatistických experimentov. 
Vzali sme na seba rolu experimentátora a jedným z našich cieľov bolo nájsť tú triedu experimentu, ktorá nám ponúkne čo najvhodnejšie výsledky.

V praktickej časti sme najprv identifikovali tie návrhy, v ktorých je možné odhadnúť žiadanú veličinu - kontrast medzi dvoma možnými ošetreniami. 
Následne sme určili, v ktorých návrhoch vieme tento kontrast odhadnúť čo najpresnejšie, a vytvorili sme jednoduchý algoritmus, ako tieto návrhy zostrojiť pre ľubovoľný rozmer. 
Algoritmus napísaný v jazyku \emph{python} sa nachádza v prílohe \hyperref[appendix:d]{D}.

Hoci sme ponúkli niekoľko zaujímavých záverov, priestor na bádanie sme zďaleka nevyčerpali. 
Pripomíname, že sme sa obmedzili na veľmi konkrétnu triedu experimentov. Mechanizmy, ktoré sme pri výskume použili, však majú potenciál byť využité aj pri iných druhoch experimentov.

Mohli by sme napríklad pracovať nie s dvoma typmi ošetrení, ako tomu bolo v našej práci, ale s troma a viac. 
Ďalším z možných rozšírení by bolo “zakázanie” niektorých kombinácií dvoch faktorov a jeho vplyv na optimálny návrh experimentu.

Jedná sa o problematiku, ktorej nebolo venované toľko článkov a publikácií, ako iným, populárnejším témam. 
Nech aj to je pre nás motiváciou k ďalšiemu bádaniu.