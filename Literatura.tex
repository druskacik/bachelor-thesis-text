\begin{thebibliography}{99}

	\bibitem{kniha} Cooper, W. W., Seiford, L. M., Tone, K.: {\it Data Envelopment Analysis A Comprehensive Text with Models, Applications, References and DEA-Solver Software}, Kluwer Academic Publishers,  Boston, 2000
	
	\bibitem{skripta} Siebertová, Z.: {\it Prednášky z ekonometrie}, učebné texty, FMFI UK Bratislava, 2007, dostupné na internete (xx.xx.2017 {\tt - vložiť dátum stiahnutia z internetu}):  http://www.defm.fmph.uniba.sk/ludia/siebertova/ekonometria2011.html
	
	\bibitem{clanok} Ševčovič D., Halická M., Brunovský, P.: {\it DEA analysis for a large structured bank branch network}, Central European Journal of Operational Research 9 (2001), 329--342, dostupné na internete (xx.xx.2017  {\tt - vložiť dátum stiahnutia z internetu}):  http://www.iam.fmph.uniba.sk/institute/sevcovic/papers/cl19.pdf
\end{thebibliography}
%
%
%Zoznam použitej literatúry obsahuje úplný zoznam bibliografických odkazov. Rozsah tejto časti je daný množstvom použitých literárnych zdrojov, ktoré musia korešpondovať s citáciami použitými v texte.
%
%Jednotlivé položky v zozname bibliografických odkazov sa uvádzajú v abecednom poradí. Sú usporiadané podľa prvého prvku (údaja), za ktorým nasleduje rok vydania dokumentu. Za nim v prípade potreby nasledujú malé písmená, ktorými sa odlišujú odkazy s rovnakým prvým údajom a rokom vydania.  {\it V matematických prácach je zaužívaný iný spôsob citovania (tzv. modifikovaná metóda číselných odkazov), kde práce usporiadame podľa abecedy a každej práci priradíme poradové číslo uvedené v hranatých zátvorkách. Odkazujeme potom napr.  na prácu [7].}
%
%Pri citovaní je dôležitá etika citovania ako aj technika citovania. Etika citovania určuje spôsob dodržiavania etickej normy vo vzťahu k cudzím myšlienkam a výsledkom, ktoré sú obsiahnuté v iných dokumentoch a v použitej literatúre. Technika citovania, vyjadruje, či a ako správne, podľa normy STN ISO 690: 1998. Dokumentácia - Bibliografické odkazy - Obsah, forma a štruktúra., autor spája miesta v texte so záznamami o dokumentoch, ktoré sú v zozname bibliografických odkazov.
%
%Pri záverečných prácach odporúčame používať {\it modifikovanú metódu číselných odkazov. Možno však používať aj tu popísanú}  metódu citovania podľa prvého údaja (mena) a dátumu, pri ktorej sa v texte uvedie v zátvorkách prvý údaj (priezvisko autora, alebo prvé slovo z názvu) a rok vydania citovaného dokumentu. Ak sa prvý údaj už nachádza v rámci textu, v zátvorkách za nim sa uvedie len rok. V prípade potreby sa v zátvorkách uvedú za rokom aj čísla citovaných strán. Ak majú dva alebo niekoľko dokumentov ten istý prvý údaj a rovnaký rok, odlíšia sa malými písmenami (a, b, c, a pod.) za rokom vo vnútri zátvoriek. To isté sa urobí aj v zozname bibliografických odkazov.
%
%Príklady popisu dokumentov citácií podľa ISO 690 a ISO 690-2:
%
%V \TeX-u používame prostredie 
%\begin{verbatim}
%\begin{thebibliography}{99}
%  \bibitem{meno autora} ...
%  \bibitem{meno autora} ...
%	\bibitem{meno autora} ...
%\end{thebibliography}
%\end{verbatim}
%
%Podľa typu citovanej literatúry ju za \texttt{$\backslash$bibitem{meno autora}} píšeme v nasledovnom tvare:
%
%\subsection*{Knihy / Monografie} 
%Prvky popisu: \\
%Autor. rok vydania. Názov: podnázov (nepovinný). Poradie vydania. Miesto vydania: Vydavateľ, rok vydania. Rozsah strán. ISBN. 
%
%Ak sú traja autori oddeľujú sa pomlčkou. Ak je viac autorov ako traja uvedie sa prvý autor a skratka a kol. alebo et al. ak je to zahraničné dielo. Prvé vydanie sa v citačnom popise nemusí uvádzať.
%
%Príklady: 
%\begin{itemize}
%  \item OBERT, V. 2006. Návraty a odkazy. Nitra : Univerzita Konštantína Filozofa, 2006. 129 s. ISBN 80-8094-046-0.
%  \item{timko} TIMKO, J. - SIEKEL. P. - TURŇA. J. 2004. Geneticky modifikované organizmy. Bratislava: Veda, 2004. 104 s. ISBN 80-224-0834-4.
%	\item HORVÁT, J. a kol. 1999. Anatómia a biológia človeka. 1. vyd. Bratislava: Obzor, 1999. 425 s. ISBN 80-07-00031-5.
%\end{itemize}
%
%\subsection*{Článok v časopise}
%Prvky popisu: \\
%Autor. rok vydania. Názov. In Názov zdrojového dokumentu (noviny, časopisy). ISSN, rok, ročník, číslo zväzku, rozsah strán (strana od-do).
%
%Príklady:
%\begin{itemize}
%	\item STEINEROVÁ, J. 2000. Princípy formovania vzdelania v informačnej vede. In {\it Pedagogická revue}. ISSN 1335-1982, 2000, roč. 2, č. 3, s. 8-16.
%	\item BEŇAČKA, J. et al. 2009 A better cosine approximate solution to pendulum equation. In {\it International Journal of Mathematical Education in Science and Technology}. ISSN 0020-739X, 2009, vol. 40, no. 2, p. 206-215. 
%\end{itemize}
%
%\subsection*{Článok zo zborníka a monografie}
%Prvky popisu: \\
%Autor. rok vydania. Názov článku. In {\it Názov zborníka}. Miesto vydania : Vydavateľ, rok vydania. ISBN, Rozsah strán (strana od-do). 
%
%Príklady:
%\begin{itemize}
%	\item ZEMÁNEK, P. 2001 The machines for ''green works'' in vineyards and their economical evaluation. In {\it 9th International Conference: proceedings}. Vol. 2. {\it Fruit Growing and viticulture}. Lednice : Mendel University of Agriculture and Forestry, 2001. ISBN 80-7157-524-0, p. 262-268.
%	\item BOĎOVÁ, M. et al. 1990. An introduction to algorithmic and cognitive approaches for information retrieval. In 18. Informatické dni: sborník referátů z mezinárodní vědecké konference o současných poznatcích informačních a komunikačních technologiích a jejich využití. Praha : Univerzita Karlova, 1990. ISBN 80-01-02079-7. s. 17-28.
%\end{itemize}
%
%
%\subsection*{Elektronické dokumenty - monografie}
%Prvky popisu: \\
%Autor. rok vydania. {\it Názov} [Druh nosiča]. Vydanie. Miesto vydania: Vydavateľ, dátum vydania. Dátum aktualizácie. [Dátum citovania]. Dostupnosť a prístup. ISBN.
%
%Príklad:
%\begin{itemize}
%	\item SPEIGHT, J. G. 2005. {\it Lange's Handbook of Chemistry}. [online]. London : McGraw-Hill, 2005. 1572 p. [cit. 2009.06.10.] Dostupné na internete:  <http://www.knovel.com/web/portal/basic\_search/display?\_EXT\_KNOVEL\\
%	\_DISPLAY\_bookid$=$1347\&\_EXT\_KNOVEL\_DISPLAY\_fromSearch$=$true\&\_ \\
%	EXT\_KNOVEL\_DISPLAY\_searchType$=$basic>. ISBN 978-1-60119-261-5.
%\end{itemize}
%	
%\subsection*{Články v elektronických časopisoch a iné príspevky}
%Prvky popisu: \\
%Autor. rok vydania. Názov. In {\it Názov časopisu}. [Druh nosiča]. Rok vydania, ročník, číslo [dátum citovania]. Dostupnosť a prístup. ISSN. 
%
%Príklad:
%\begin{itemize}
%	\item HOGGAN, D. Challenges, Strategies, and Tools for Research Scientists. In {\it Electronic Journal of Academic and Special Librarianship} [online]. 2002, vol. 3, no. 3 [cit. 2003-01-10]. Dostupné na internete: <http://southernlibrarianship.icaap.org/content/v03n03/Hoggan\_d01.htm>. ISSN 1525-321X.
%\end{itemize}
%
%\subsection*{Príspevok v zborníku na CD-ROM}
%Prvky popisu: \\
%Autor. rok vydania. Názov. In {\it Názov zborníka} [Druh nosiča]. Miesto vydania : Vydavateľ, rok vydania,  rozsah strán (strana od-do). ISBN.
%
%Príklad:
%\begin{itemize}
% 	\item ZEMÁNEK, P. The machines for ''green works'' in vineyards and their economical evaluation. In {\it 9th International Conference: proceedings}. Vol. 2. Fruit Growing and viticulture [CD-ROM]. Lednice : Mendel University of Agriculture and Forestry, 2001, p. 262-268. ISBN 80-7157-524-0.
%\end{itemize}
%
%\subsection*{Vedecko-kvalifikačné práce}
%Prvky popisu: \\
%Autor. rok vydania. Názov práce : označenie druhu práce (dizertačná, doktorandská). Miesto vydania : Názov vysokej školy. Rok vydania. Rozsah strán.
%
%Príklad:
%\begin{itemize}
%	\item MIKULÁŠIKOVÁ, M. 1999. Didaktické pomôcka pre praktickú výučbu na hodinách výtvarnej výchovy pre 2. stupeň základných škôl : diplomová práca. Nitra: UKF, 1999. 62 s.
%\end{itemize}
%
%
%\subsection*{Výskumné správy}
%Prvky popisu: \\
%Autor. Rok vydania. {\it Názov práce} : druh správy (VEGA, priebežná správa). Miesto vydania : Názov inštitúcie, rok vydania. Rozsah strán.
%
%Príklad:
%\begin{itemize}
%	\item BAUMGARTNER, J. a kol. 1998 {\it Ochrana a udržiavanie genofondu zvierat, šľachtenie zvierat}: výskumná správa. Nitra : VÚŽV, 1998. 78 s.
%\end{itemize}
%
%\subsection*{Normy}
%Popis prvku: \\
%Označenie a číslo normy. Rok vydania (nie rok schválenia, alebo účinnosti): Názov normy.
%
%Príklad:
%\begin{itemize}
%	\item STN ISO 690:1998 Dokumentácia - Bibliografické odkazy - Obsah, forma a štruktúra.
%\end{itemize}